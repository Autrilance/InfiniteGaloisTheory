\chapter{Profinite Group}

\begin{definition}
	\lean{ProfiniteGrp}
	\leanok
	A profinite group is a topological group that is compact and totally disconnected.
\end{definition}

\begin{lemma}
	\lean{ProfiniteGrp.ofContinuousMulEquivProfiniteGrp}
	\leanok
	If $G$ is a profinite topological group, $H$ is a topological group that is isomorphic to $G$, then $G$ is profinite.
\end{lemma}

\begin{lemma}
	Product of (arbitrarily many) profinite groups is profinite.
\end{lemma}

\begin{lemma}
	\lean{ProfiniteGrp.ofClosedSubgroup}
	\leanok
	If $G$ is a profinite topological group, then any closed subgroup of $G$ is profinite.
\end{lemma}

\begin{theorem}
	If $G$ is a profinite topological group, $H$ is a subgroup of $G$, then the following three statements
	are equivalent.
	\begin{itemize}
		\item[(i)] $ H $ is closed.
		\item[(ii)] $ H $ is profinite.
		\item[(iii)] $ H $ is the intersection of a family of open subgroups.
	\end{itemize}
\end{theorem}

\begin{lemma}
	If $G$ is a profinite topological group, $H$ is a closed subgroup of $G$, then the homogeneous space $G / H$ is compact and totally disconnected. Moreover, if $H$ is normal, then $G / H$ is a profinite topological group.
\end{lemma}

\begin{lemma}
	Let $ G $ be a profinite group, then any open subgroup of $ G $ must has finite index, thus a subgroup of $ G $ is open if and only if it is closed and has finite index.
\end{lemma}

\begin{lemma}
	Let $ G $ be a profinite group, then every neighborhood of the identity contains an open normal subgroup.
\end{lemma}

\begin{lemma}
	Let $ G $ be a profinite group, then open normal subgroups of the identity form a neighborhood basis of $ G $.
\end{lemma}

\begin{theorem}
	%\lean{ProfiniteGrp.instHasLimitCompFiniteGrpFromFiniteGrp}
	A topological group is profinite if and only if it is the projective limit of finite groups, each given the discrete topology.
\end{theorem}
